\documentclass[12pt]{article}
\usepackage[utf8]{inputenc}
\usepackage{parskip}

\usepackage[bottom]{footmisc}

\usepackage{graphicx}

\usepackage{subcaption}


\begin{document}

\section*{(Parte 2) LLenalo con super}

\subsection*{Presentación del problema}
Dado un vendedor que cuenta con vehiculo propio que necesita moverse entre ciudades personalmente para vender sus productos, quiere buscar la manera más económica de realizar la tarea.

Teniendo en cuenta que las distancias entre ciudades estan representadas en kilometros y el precio de la nafta es por litro queremos minimizar el costo en nafta de llegar de cada ciudad a las demas teniendo en cuenta las siguientes propiedades: 

\begin{enumerate}
    \item Las rutas que comunican las ciudades son bidireccionales.
    \item El precio de la nafta varia de ciudad en ciudad.
    \item El auto posee un limite de litros de nafta que puede cargar (60 litros).
    \item Se estima que cada litro de nafta alcanza para exactamente un kilometro de distancia.
\end{enumerate}

El problema formalmente será representado como un grafo donde cada ciudad
 $c_i$ es un vertice con su respectivo precio por litro $p_i$ y donde 
 cada distancia $d_{x,y}$ que comunica un par de ciudades ($c_x, c_y$) son aristas.

Representado el $r_k$ como un camino posible entre dos ciudades en función del costo efectivo
comprendido como la cantidad de litros $l_i$ que cargue de nafta por su respectivo precio $p_i$ queremos los caminos que cumplen que:

$\forall(c_x, c_y \in Ciudades)(r_{m}, r_k \in Caminos(c_x,c_y))\rightarrow(r_{m} < r_k)$

Es decir nos interesan todos los caminos minimos.

Una motivación razonable para resolver el problema sería entonces simplemente aplicar algoritmos conocidos de camino minimo, sin embargo esto no es posible.

Para empezar el precio al estar en función de los vertices, los costes no dependen solo de las distancias, sino de una multiplicación entre litros cargados por el precio.

Las rutas son bidireccionales, sin embargo dadas 2 ciudades $C_a$ y $C_b$ no se cumple que los costes $C_a \rightarrow C_b$ y $C_b \rightarrow C_a$ sean iguales.
Por ejemplo:

\begin{figure}[h!]
    \centering
    \begin{subfigure}[b]{0.4\linewidth}
      \includegraphics[width=\linewidth]{{graficos/caminoSimple}.pdf}
      \caption{$fig_1$ bidireccional.pdf}
    \end{subfigure}
    \begin{subfigure}[b]{0.4\linewidth}
      \includegraphics[width=\linewidth]{{graficos/caminoSimple2}.pdf}
      \caption{$fig_2$ direccional.pdf}
    \end{subfigure}
\end{figure}

En particular $\forall(p_a \neq p_b) \rightarrow (p_a * d_{a,b} \neq p_b * d_{a,b}) $

Además los caminos minimos no necesariamente son simples, por ejemplo en $fig.1$ se pueden observar 3 ciudades

\begin{figure}
    \includegraphics[width=0.7\textwidth]{{graficos/caminoNoSimple}.pdf}
\end{figure}




\end{document}