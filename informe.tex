\documentclass[12pt]{article}
\usepackage[utf8]{inputenc}
\usepackage{parskip}

\usepackage[bottom]{footmisc}

\usepackage{graphicx}

\usepackage{subcaption}


\begin{document}

\section{Segmentation is my fault}
\subsection{Introducción y presentación del trabajo realizado}

La segmentación de imágenes de forma es un problema abierto en el área de
\emph{visión artificial} que dada la variedad de sus aplicaciones posee varios
intereses en conflicto al querer desarrollar un algoritmo definitivo para ésta.
Un algoritmo eficiente que requiera tiempo y memoria lineal puede ser útil para
procesamiento de video en tiempo real sacrificando la calida del resultado,
mientras que otro enfoque más costoso puede ser útil a la hora de ofrecer una
herramienta que facilite labores artísticos (separación automática de los
objetos en una escena, por ejemplo).

\begin{figure}[h]
	\centering
	\begin{subfigure}{0.4\linewidth}
		\includegraphics[width=\linewidth]{segmentation/entradas-posta/oso}
		\caption{Entrada}
	\end{subfigure}
	\begin{subfigure}{0.4\linewidth}
		\includegraphics[width=\linewidth]{segmentation/salidas/{0.8.oso.jpg.500.100}.png}
		\caption{Segmentación}
	\end{subfigure}
	\caption{$\sigma = 0.8,\ k = 500,\ g = 100$.}
\end{figure}

El algoritmo a implementar es el propuesto por \textbf{Pedro F.  Felzenszwalb}
y \textbf{Daniel P. Huttenlocher} en su trabajo \textsc{Efficient Graph-Based
Image Segmentation}. Éste algoritmo construye un grafo grilla con cada píxel en
su propio segmento y luego uno los segmentos golosamente dada una condición de
``similaridad suficente'' entre ellos. Para mejorar los resultados del
algoritmo se implementaron dos pasos de pre y post procesamiento.

El preprocesamiento consiste de un desenfoque gaussiano de dos pasadas. Éste se
implementó dentro del programa entregado luego de notar que desenfocar la
imagen y luego guardarla en el formato de entrada propuesto por la cátedra
resultaba en artefactos indeseables propios del bajo rango de valores de cada
pixel. El sigma de desenfoque puede ser modificando variando el segundo
parámetro pasado al ejecutable.

El postprocesamiento es una segunda pasada por las aristas del grafo forzando
la unión de todos los componentes suficientemente chicos. El significado de
``suficientemente chico'' se puede modificar variando el tercer parámetro
pasado al ejecutable. El parámetro es relativo a la resolución de la imagen a
procesar, un valor de $1$ eliminará todas las componentes que posean menos del
$100\%$ de los píxeles, uno de $10$ todas las que posean menos del $10\%$.
Llamaremos $g$ a este parámetro dado que cuanto mayor es su valor mayor es la
\emph{granularidad} de la segmentación resultante\footnote{Efectivamente el
parámetro pone un límite superior a la cantidad de segmentos que puede tener la
segmentación.}.

El algoritmo en sí posee un valor $k$ que representa la ``soltura'' a la hora
de comparar componentes para decidir si unirlas o no. Éste parámetro es el
primero que recibe el ejecutable entregado.

La elección de colores para las segmentaciones mostradas en este informe se
realizó aprovechándose de detalles implementativos. Esta elección nos permite
que los colores sean estables al variar tanto $k$ cómo $g$, lo cuál permitió
crear los videos adjuntos a este informe.

\subsection{Presentación informal e intuitiva del algoritmo}

El algoritmo implementado construye un \emph{grafo grilla} a partir de la
imagen proporcionada y luego recorre las aristas del mismo de menor a mayor
uniendo las compomentes de cada lado de la arista. La motivación es simple,
procesar los píxeles más similares al principio nos forzará a recorrer las
regiones de la imagen desde las de menos variabilidad hasta las que sean
básicamente ruido blanco.

Para tomar la decisión de si unir o no dos componentes se propone calcular la
\emph{diferencia interna} de una componente de la siguiente forma:

$$Int(C) = \max_{a \in AGM(C_V, C_E)} p(a)$$

Dónde $AGM(V, E)$ es el árbol generador mínimo del subgrafo de la componente
$C$, $C_V$ son los vértices de la componente V, $C_E$ son los ejes de la
componente $C$ y $p(e)$ es la función que determina el peso de una
arista\footnote{ Dependiendo el espacio de color de la imagen y la forma en la
que se representen los mismos puede haber muchas funciones que tenga sentido
evaluar como $p$.}.

\begin{figure}[h]
	\centering
	\includegraphics[width=0.9\linewidth]{graficos/grilla}
	\caption{Grafo grilla con 8-vecindad.}
\end{figure}

Dado todo esto, dos componentes que limiten en un par de píxeles se unirán si
la diferencia en el límite está dentro de la tolerancia común de las mismas, es
decir si $\textit{diff} \geq Int(C_1) + \tau(C_1) \land \textit{diff} \geq
Int(C_2) + \tau(C_2)$, dónde $\tau(C)$ es la función que le da juego\footnote{
En el sentido de que suma cierta holgura que permite unir componentes de que
otro parecerían demasiado distintas, cosa que es común con componentes
pequeñas} a la componente. La función propuesta para $\tau(C)$ es la siguiente:

$$\tau(C) = \frac{K}{|C_V|}$$

La que nos ofrece un valor cercano a $K$ para compontentes muy chicas y uno
cercano a $0$ para componentes muy grandes. Ésto fuerza a tener evidencias
claras de que dos componentes son en realidad la misma al ser estas muy grandes
pero nos permite un montón de juego en las primeras etapas.

\subsubsection{Entonces. ¿Cuál es el algoritmo aquí propuesto?}

Construiremos una primera aproximación burda (ubicando cada píxel en su propia
componente) y lo refinaremos preguntándonos si dado un par de componentes
unidas por un eje éstas no deberían en realidad unirse. A la hora de ejecutar
un algorimo así de goloso es importante el orden en el cuál se toman las
aristas a procesar el obvio (y el utilizado por nosotros) es de menor a mayor
peso\footnote{ En una implementación que requiera un tiempo de procesamiento
lineal se puede utilizar un \emph{Counting Sort} sabiendo cuál es la mayor
diferencia posible (256 en el caso de la entrada de la cátedra). Sacrificando
cierta precisión numérica claro está.}.

TODO: Contar con fotitos qué hace el algoritmo


\section{LLenalo con super}

\subsection*{Presentación del problema}
Dado un vendedor que cuenta con vehiculo propio que necesita moverse entre ciudades personalmente para vender sus productos, quiere buscar la manera más económica de realizar la tarea.

Teniendo en cuenta que las distancias entre ciudades estan representadas en kilometros y el precio de la nafta es por litro queremos minimizar el costo en nafta de llegar de cada ciudad a las demas teniendo en cuenta las siguientes propiedades: 

\begin{enumerate}
    \item Las rutas que comunican las ciudades son bidireccionales.
    \item El precio de la nafta varia de ciudad en ciudad.
    \item El auto posee un limite de litros de nafta que puede cargar (60 litros).
    \item Se estima que cada litro de nafta alcanza para exactamente un kilometro de distancia.
\end{enumerate}

El problema formalmente será representado como un grafo donde cada ciudad
 $c_i$ es un vertice con su respectivo precio por litro $p_i$ y donde 
 cada distancia $d_{x,y}$ que comunica un par de ciudades ($c_x, c_y$) son aristas.

Representado el $r_k$ como un camino posible entre dos ciudades en función del costo efectivo
comprendido como la cantidad de litros $l_i$ que cargue de nafta por su respectivo precio $p_i$ queremos los caminos que cumplen que:

$\forall(c_x, c_y \in Ciudades)(r_{m}, r_k \in Caminos(c_x,c_y))\rightarrow(r_{m} < r_k)$

Es decir nos interesan todos los caminos minimos.

Una motivación razonable para resolver el problema sería entonces simplemente aplicar algoritmos conocidos de camino minimo, sin embargo esto no es posible.

Para empezar el precio al estar en función de los vertices, los costes no dependen solo de las distancias, sino de una multiplicación entre litros cargados por el precio.

Las rutas son bidireccionales, sin embargo dadas 2 ciudades $C_a$ y $C_b$ no se cumple que los costes $C_a \rightarrow C_b$ y $C_b \rightarrow C_a$ sean iguales.
Por ejemplo:

\begin{figure}[h!]
    \centering
    \begin{subfigure}[b]{0.4\linewidth}
      \includegraphics[width=\linewidth]{{graficos/caminoSimple}.pdf}
      \caption{$fig_1$ bidireccional.pdf}
    \end{subfigure}
    \begin{subfigure}[b]{0.4\linewidth}
      \includegraphics[width=\linewidth]{{graficos/caminoSimple2}.pdf}
      \caption{$fig_2$ direccional.pdf}
    \end{subfigure}
\end{figure}

En particular $\forall(p_a \neq p_b) \rightarrow (p_a * d_{a,b} \neq p_b * d_{a,b}) $

Además los caminos minimos no necesariamente son simples, por ejemplo en $fig.1$ se pueden observar 3 ciudades

\begin{figure}
    \includegraphics[width=0.7\textwidth]{{graficos/caminoNoSimple}.pdf}
\end{figure}




\end{document}
