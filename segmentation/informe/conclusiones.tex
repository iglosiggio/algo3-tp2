\subsection{Conclusiones y trabajo futuro}

Con lo expuesto anteriormente está claro que el resultado de esta técnica es
muy sensible a los parámetros dados. Buscar heurísticas para encontrar éstos es
un camino posible para facilitar el uso del algoritmo así cómo otras
formulaciones para la diferencia entre píxeles y para $\tau(C)$. En el trabajo
original se muestra que utilizar percentiles de las componentes para calcular
$\tau(C)$ pone al problema en la clase \emph{NP-Hard}.

Otro punto en el cual creemos que faltó trabajo es en utilizar diferentes
imágenes para ver cómo afectan al tiempo dadas las distintas implementaciones
de \textsc{Disjoint-Set}.
