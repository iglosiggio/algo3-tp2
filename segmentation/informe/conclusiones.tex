\subsection{Conclusiones y trabajo futuro}

Con lo expuesto anteriormente está claro que el resultado de esta técnica es
muy sensible a los parámetros dados y a la claridad de la imagen de entrada.
Buscar heurísticas para encontrar éstos es un camino posible para facilitar el
uso del algoritmo así cómo otras formulaciones para la diferencia entre píxeles
y para $\tau(C)$.

A pesar de la sensibilidad de los parámetros el implementar el desenfoque
gaussiano mostró resultados cualitativos satisfactorios. La razón por la que
creamos dicho filtro fue que realizar el desenfoque por fuera del programa
introducia errores de quantización al guardar el resultado intermedio en un
entero con rango acotado (0 a 255).

Pese a que utilizamos diferentes imágenes para ver cómo afectan al tiempo dadas
las distintas implementaciones consideramos que estos casos fueron explorados
con poco detalle. También nos hubiese gustado realizar gráficos de correlación
para comparar las complejidades teóricas con los tiempos mediod.  Dicha tarea
no fue abordada por escasez de tiempo y consideramos un faltante que nos quedó
pendiente.

Creemos también que los videos que complementan éste informe resultan útiles
para entender las cualidades de la técnica. Pero dado que la selección de
colores no es estable respecto de $\sigma$ creemos que el vídeo en el que lo
variamos no aporta la intuición que esperábamos.

A pesar de todo, consideramos satisfactorios los resultados obtenidos, se puede
distinguir objetos, figuras, detalles en las imagenes con diferentes grados de
precisión, para cualquier tamaño. Así mismo notamos que los parametros $k =
[300, 600]$, $\sigma = 0.8$ y $g = 1000$ han mostrado buenos resultados en
líneas generales.
