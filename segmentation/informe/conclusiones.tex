\subsection{Conclusiones y trabajo futuro}

Con lo expuesto anteriormente está claro que el resultado de esta técnica es
muy sensible a los parámetros dados y a la claridad de la imagen de entrada.
Buscar heurísticas para encontrar éstos es un camino posible para facilitar el
uso del algoritmo así cómo otras formulaciones para la diferencia entre píxeles
y para $\tau(C)$.

Otro punto en el cual creemos que faltó trabajo es en utilizar diferentes
imágenes para ver cómo afectan al tiempo dadas las distintas implementaciones.
de \textsc{Disjoint-Set}. Creemos también que los videos que complementan éste
informe resultan útiles para entender las cualidades de la técnica. Pero dado
que la selección de colores no es estable respecto de $\tau$ una estrategia
mejor para construir las visualizaciones variando los parámetros es algo que
creemos nos faltó.
