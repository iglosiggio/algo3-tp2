\subsection{Introducción y presentación del trabajo realizado}

La segmentación de imágenes de forma es un problema abierto en el área de
\emph{visión artificial} que dada la variedad de sus aplicaciones posee varios
intereses en conflicto al querer desarrollar un algoritmo definitivo para ésta.
Un algoritmo eficiente que requiera tiempo y memoria lineal puede ser útil para
procesamiento de video en tiempo real sacrificando la calida del resultado,
mientras que otro enfoque más costoso puede ser útil a la hora de ofrecer una
herramienta que facilite labores artísticos (separación automática de los
objetos en una escena, por ejemplo).

\begin{figure}[h]
	\centering
	\begin{subfigure}{0.4\linewidth}
		\includegraphics[width=\linewidth]{segmentation/entradas-posta/oso}
		\caption{Entrada}
	\end{subfigure}
	\begin{subfigure}{0.4\linewidth}
		\includegraphics[width=\linewidth]{segmentation/salidas/{0.8.oso.jpg.500.100}.png}
		\caption{Segmentación}
	\end{subfigure}
	\caption{$\sigma = 0.8,\ k = 500,\ g = 100$.}
\end{figure}

El algoritmo a implementar es el propuesto por \textbf{Pedro F.  Felzenszwalb}
y \textbf{Daniel P. Huttenlocher} en su trabajo \textsc{Efficient Graph-Based
Image Segmentation}. Éste algoritmo construye un grafo grilla con cada píxel en
su propio segmento y luego uno los segmentos golosamente dada una condición de
``similaridad suficente'' entre ellos. Para mejorar los resultados del
algoritmo se implementaron dos pasos de pre y post procesamiento.

El preprocesamiento consiste de un desenfoque gaussiano de dos pasadas. Éste se
implementó dentro del programa entregado luego de notar que desenfocar la
imagen y luego guardarla en el formato de entrada propuesto por la cátedra
resultaba en artefactos indeseables propios del bajo rango de valores de cada
pixel. El sigma de desenfoque puede ser modificando variando el segundo
parámetro pasado al ejecutable.

El postprocesamiento es una segunda pasada por las aristas del grafo forzando
la unión de todos los componentes suficientemente chicos. El significado de
``suficientemente chico'' se puede modificar variando el tercer parámetro
pasado al ejecutable. El parámetro es relativo a la resolución de la imagen a
procesar, un valor de $1$ eliminará todas las componentes que posean menos del
$100\%$ de los píxeles, uno de $10$ todas las que posean menos del $10\%$.
Llamaremos $g$ a este parámetro dado que cuanto mayor es su valor mayor es la
\emph{granularidad} de la segmentación resultante\footnote{Efectivamente el
parámetro pone un límite superior a la cantidad de segmentos que puede tener la
segmentación.}.

El algoritmo en sí posee un valor $k$ que representa la ``soltura'' a la hora
de comparar componentes para decidir si unirlas o no. Éste parámetro es el
primero que recibe el ejecutable entregado.

La elección de colores para las segmentaciones mostradas en este informe se
realizó aprovechándose de detalles implementativos. Esta elección nos permite
que los colores sean estables al variar tanto $k$ cómo $g$, lo cuál permitió
crear los videos adjuntos a este informe.
