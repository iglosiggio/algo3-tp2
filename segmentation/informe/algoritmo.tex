\subsection{El algoritmo (y sus partes)}

En esta sección mostraremos el pseudocódigo correspondiente a este trabajo en
conjunto con información sobre detalles implementativos.

Los algoritmos acá mostrados usan todos alguna implementación de la estructura
\textsc{Disjoint-Set} que posee los siguientes observadores y operaciones:

\begin{itemize}
	\item \Call{Nuevo}{$s$}
		--- Retorna un nuevo \textsc{Disjoint-Set} con $s$ elementos.
	\item \Call{Buscar}{ds, $e$}
		--- Retorna el identificador de la componente de $e$.
	\item \Call{Unir}{ds, $a$, $b$}
		--- Une al componente que contiene a $a$ con el componente que
		contiene a $b$.
	\item \Call{Tamaño}{ds, $e$}
		--- Retorna el tamaño (en cantidad de elementos) de la
		componente que contiene a $e$.
\end{itemize}

\subsubsection{Desenfoque}

El desenfoque aplicado es un desenfoque gaussiano. En líneas generales la
fórmula a aplicar es la siguiente:

\[
	f(x,y,\sigma) =
		\sum_{j=0}^{alto} \! \sum_{i=0}^{ancho} \!
			\frac{1}{2\pi\sigma^2}
			e^{-\frac{{(x - j)}^2 + {(y - i)}^2}{2\sigma^2}}
			\textit{imagen}[j][i]
\]

Puede observarse que los píxeles más lejanos al pixel que queremos calcular
aportan notablemente menos, por esto es que cuando implementamos el desenfoque
limitamos la cantidad de vecinos a tener en cuenta. Otra optimización simple
nace de precalcular los exponentes de $e$ dado que si sólo tenemos en cuenta
los $k$ píxeles más cercanos entonces sólo operaremos con $k$ exponentes
distintos\footnote{En la implementación esto se achica aún más dado que el
filtro es separable (puede transformarse en una pasada vertical y una
horizonal)}.

La implementación del desenfoque se realiza generando una matriz de acuerdo al
$\sigma$. Limitamos la cantidad de términos de la normal a calcular a los
naturales desde $0$ hasta $\lceil 4 \sigma \rceil$ (cuando necesitamos valores
negativos de la normal nos aprovechamos de la simetría de la misma). El tope
que escojemos hace que los coeficientes que quedan fuera sean menores a
$e^{-8}$ que son suficentemente pequeños cómo para no resultarnos relevantes.
Nos aprovecharnos de la \emph{separabilidad} del filtro realizando una
convolución en cada eje en lugar de tener que usar la matriz completa.

\begin{algorithm}[H]
\caption{Algoritmo para realizar un desenfoque gaussiano}
\begin{algorithmic}[1]
\Statex{}
\Function{Desenfocar}{$w$, $h$, imagen, $\sigma$}`
	\Statex{} \Comment{} Por simpleza no están los chequeos de límites en la imagen
	\State{} $\text{kernel} \is []$
	\For{$i \is 0$ to $\lceil 4 \sigma \rceil + 1$}
	\Comment{Coeficientes de la normal a usar}
		\State{} $\text{kernel}[i]
			\is \frac{1}{\sqrt{2\pi\sigma^2}}e^{-\frac{i^2}{2\sigma^2}}$
	\EndFor{}
	\For{$y \is 0$ to $h$}
		\Comment{} Convolución horizontal
		\For{$x \is 0$ to $w$}
			\State{} $\text{sum} \is 0$
			\For{$i \is 0$ to $|\text{kernel}|$}
				\State{} $\text{sum} \is \text{sum}
					+ \text{kernel}[i] \times (
						\text{imagen}[y][x - 1]
						+ \text{imagen}[y][x + 1]
					)$
			\EndFor{}
			\State{} $\text{imagen}[y][x] \is \text{sum}$
		\EndFor{}
	\EndFor{}
	\For{$y \is 0$ to $h$}
		\Comment{} Convolución vertical
		\For{$x \is 0$ to $w$}
			\State{} $\text{sum} \is 0$
			\For{$i \is 0$ to $|\text{kernel}|$}
				\State{} $\text{sum} \is \text{sum}
					+ \text{kernel}[i] \times (
						\text{imagen}[y - 1][x]
						+ \text{imagen}[y + 1][x]
					)$
			\EndFor{}
			\State{} $\text{imagen}[y][x] \is \text{sum}$
		\EndFor{}
	\EndFor{}
\EndFunction{}
\end{algorithmic}
\end{algorithm}

Puede observarse que la construcción del kernel nos llevará $O(\sigma)$
operaciones seguidas de $O(w \times h \times \sigma)$ para cada una de las
convoluciones. Entonces el costo del desenfoque es $O(w \times h \times
\sigma)$ (lineal respecto del tamaño de la imagen, exponencial respecto del
tamaño de $\sigma$).

\subsubsection{Construcción del grafo}

Para construir el grafo utilizamos la 8-vecindad cómo ya dijimos anteriormente.
Una vez cargada la imagen podemos recorrer todos los píxeles (que serán
vértices de nuestro grafo) y agregar los ejes necesarios.

\begin{algorithm}[H]
\caption{Algoritmo para construir el grafo a segmentar}
\begin{algorithmic}[1]
\Statex{}
\Function{Construir-Grafo}{$w$, $h$, imagen}
	\Statex{} \Comment{} Describimos a los ejes como \{desde, hasta, peso\}
	\State{} $\text{ejes} \is \emptyset$
	\For{$y \is 0$ to $h$}
		\For{$x \is 0$ to $w$}
			\State{} $p_o \is \{x, y\}$
			\State{} $p_e \is \{x + 1, y\}$
			\If{$p_e$ está en los límites de la imagen}
				\State{} $\text{dif}
					\is \Call{Diferencia}{p_o,\ p_e,\ \text{imagen}}$
				\State{} \Call{Agregar}{ejes, \{$p_o$, $p_e$, dif\}}
			\EndIf{}
			\State{} $p_{sw} \is \{x - 1, y + 1\}$
			\If{$p_{sw}$ está en los límites de la imagen}
				\State{} $\text{dif}
					\is \Call{Diferencia}{p_o,\ p_{sw},\ \text{imagen}}$
				\State{} \Call{Agregar}{ejes, \{$p_o$, $p_{sw}$, dif\}}
			\EndIf{}
			\State{} $p_s \is \{x, y + 1\}$
			\If{$p_s$ está en los límites de la imagen}
				\State{} $\text{dif}
					\is \Call{Diferencia}{p_o,\ p_s,\ \text{imagen}}$
				\State{} \Call{Agregar}{ejes, \{$p_o$, $p_s$, dif\}}
			\EndIf{}
			\State{} $p_{se} \is \{x + 1, y + 1\}$
			\If{$p_{se}$ está en los límites de la imagen}
				\State{} $\text{dif}
					\is \Call{Diferencia}{p_o,\ p_{se},\ \text{imagen}}$
				\State{} \Call{Agregar}{ejes, \{$p_o$, $p_{se}$, dif\}}
			\EndIf{}
		\EndFor{}
	\EndFor{}
	\State{} \Return{} ejes
\EndFunction{}
\end{algorithmic}
\end{algorithm}

Puede observarse que por cada píxel se agregan a lo sumo 4 ejes. Dada la
estructura elegida para representar el grafo (un arreglo de ejes con la memoria
reservada previamente) la inserción es $O(1)$ y por estar acotada la cantidad
de ejes por vértice entonces es $O(1 + d)$ con $d$ el costo de la función de
distancia el procesar cada píxel de la imagen. Dado que procesamos toda la
imagen la complejidad resultante es $O(w \times h \times d)$ (y cómo nuestra
función de distancia es $O(1)$ entonces tenemos un coste lineal respecto del
tamaño de la imagen).

\subsubsection{Segmentación}

Con los ejes ya ordenados el proceso de segmentar la imagen es muy simple.
$\tau(C)$ es la función que dado un identificador de componente indica el
factor de juego que le corresponde a esa componente.

\begin{algorithm}[H]
\caption{Algoritmo segmentar el grafo generado}
\begin{algorithmic}[1]
\Statex{}
\Function{Segmentar}{$w$, $h$, ejes}
	\State{} $\text{ds} \is \Call{Nuevo}{w \times h}$
	\State{} $\text{umbral} \is [\tau(0), \tau(1), \dots, \tau(w \times h - 1)]$
	\For{$\text{eje} \in \text{grafo}$ (de menor a mayor peso)}
		\State{} $C_a \is \Call{Buscar}{\text{ds},\ \text{eje}.\text{desde}}$
		\State{} $C_b \is \Call{Buscar}{\text{ds}\, \text{eje}.\text{hasta}}$
		\If{$\text{eje}.\text{peso} \leq \text{umbral}[C_a]
			\land \text{eje}.\text{peso} \leq \text{umbral}[C_b]$}
			\State{} \Call{Unir}{ds, $C_a$, $C_b$}
			\State{} $C_a \is \Call{Buscar}{\text{ds},\ C_a}$
			\State{} $\text{umbral}[C_a] \is \Call{Tamaño}{\text{ds},\ C_a}
				+ \tau(C_a)$
		\EndIf{}
	\EndFor{}
	\State{} \Return{} ds
\EndFunction{}
\end{algorithmic}
\end{algorithm}

La complejidad de este algoritmo si bien es simple requiere saber las
complejidades de \Call{Nuevo}{$s$}, \Call{Buscar}{ds, $e$} y de \Call{Unir}{ds,
$a$, $b$} las que serán $O(s)$ (para todas las implementaciones), $O(f)$ y
$O(g)$ respectivamente\footnote{En secciones posteriores discutiremos la
complejidad de éstas.}. Una vez notado esto tenemos un costo inicial de $O(w
\times h)$ para construir el arreglo de resultados de $\tau(\text{e})$ y el
\textsc{Disjoint-Set} correspondiente. Hecho esto tenemo una cantidad acotada
de llamadas a \textsc{Buscar} y \textsc{Unir} por cada eje del grafo lo que nos
arroja una complejidad final de $O(w \times h \times (f + g))$.

\subsubsection{Simplificación Greedy}

Una vez segmentada la imagen se puede realizar un paso final que elimine las
componentes más pequeñas. El recorrer las aristas de menor a mayor peso hace
que las componentes que se unifiquen siempre lo hagan por la arista más liviana
de su frontera (que es lógico suponer pertenece a la componente limítrofe más
parecida).

\begin{algorithm}[H]
\caption{Algoritmo para eliminar segmentos pequeños}
\begin{algorithmic}[1]
\Statex{}
\Function{Simplificar}{$w$, $h$, ds, ejes, $g$}
	\State{} $\text{min} \is \frac{g}{w \times h}$
	\For{$\text{eje} \in \text{ejes}$ (de menor a mayor peso)}
		\If{$\Call{Tamaño}{\text{ds, eje}_a} < \text{min}
		     \vee \Call{Tamaño}{\text{ds, eje}_b} < \text{min}$}
			\State{} $C_a \is \Call{Buscar}{\text{ds, eje}_a}$
			\State{} $C_b \is \Call{Buscar}{\text{ds, eje}_b}$
			\If{$C_a \neq C_b$}
				\State{} \Call{Unir}{ds, $C_a$, $C_b$}
			\EndIf{}
		\EndIf{}
	\EndFor{}
\EndFunction{}
\end{algorithmic}
\end{algorithm}

Bajo la misma argumentación que en el algoritmo de segmentación este paso
depende del coste de la implementación del \textsc{Disjoint-Set}. El coste de
\textsc{Tamaño} será $O(h)$ dejándonos así un coste de $O(w \times h \times (f
+ g + h))$ para este paso.

\subsubsection{Vista aérea del algoritmo}

A modo de unificar todos los pasos ya expuestos en detalle podemos resumir la
operación del algoritmo en pocas líneas:

\begin{algorithm}[H]
\caption{Algoritmo para segmentar con todos los pasos comentados}
\begin{algorithmic}[1]
\Statex{}
\Function{Simplificar}{$w$, $h$, imagen, $\sigma$, $k$, $g$}
	\State{} \Call{Desenfocar}{$w$, $h$, imagen, $\sigma$}
	\State{} $\text{ejes} \is \Call{Construir-Grafo}{w,\ h,\ \text{imagen}}$
	\State{} \Call{Ordenar}{ejes}
	\State{} $\text{segmentación} \is \Call{Segmentar}{w,\ w,\ \text{ejes}}$
	\State{} \Call{Simplificar}{$w$, $h$, segmentación, ejes, $g$}
	\State{} \Return{} segmentacion
\EndFunction{}
\end{algorithmic}
\end{algorithm}

El algoritmo de ordenamiento podría ser $O(n \log n)$ de ser uno de comparación
clásico o $O(n \times w)$ (con $w$ el tamaño de las distancias) usando un
algoritmo de ordenamiento no-comparativo como \textsc{Radix-Sort}.  Con esto en
cuenta podemos sumar las complejidades de cada paso y llegar a una de $O(w
\times h \times (\sigma + f + g + h + \log n))$ o $O(w \times h \times (\sigma
+ f + g + h + w))$ de acuerdo al ordenamiento elegido (de aquí en más
utilizaremos la cota de complejidad con ordenamiento por comparación debido a
que fué el elegido en nuestra implementación).
